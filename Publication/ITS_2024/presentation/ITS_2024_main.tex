%===============================================================================
% !TeX encoding = UTF-8
% !TeX spellcheck = ru_RU-Russian
% !TEX TS-program = latexmk
\documentclass[12pt,compress,aspectratio=169]{beamer}
%\documentclass[12pt,compress]{beamer}
%===============================================================================
\usepackage{mathtools}
\usepackage{amsmath}
\usepackage{amssymb}
\usepackage{bm}
\usepackage{tabularx}
\usepackage{multirow}
\usepackage{multicol}
\usepackage{nicefrac}
\usepackage{siunitx}
\usepackage{blindtext}

\graphicspath{{./pdf/}{./png/}{../Figures/}{./pics/}}
\usepackage[inkscapelatex=false]{svg}
\svgpath{{svg/}{../Figures/}}

% \usetheme{moloch}
% \usetheme{metropolis}
\usefonttheme[onlymath]{serif}

\usepackage{FiraSans}

% \usepackage[russian,english]{babel}
\usepackage[english,main=russian]{babel}  % https://tex.stackexchange.com/questions/86078/strange-latex-compilation-errors-triggered-by-the-number-of-lines-on-the-page

%\usepackage[]{plex-sans}
% \usetheme[progressbar=frametitle]{metropolis}
%\usetheme[progressbar=frametitle]{moloch}
\usetheme{moloch}
% 

% \setsansfont[ItalicFont={Fira Sans Light Italic},%
%                  BoldFont={Fira Sans SemiBold},%
%                  BoldItalicFont={Fira Sans Italic}]%
%                 {Fira Sans Light}%

\usepackage{appendixnumberbeamer}
\usepackage{booktabs}
\usepackage[scale=2]{ccicons}
\usepackage{pgfplots}
\usepgfplotslibrary{dateplot}

\usepackage{xspace}
% \newcommand{\themename}{\textbf{\textsc{metropolis}}\xspace}

\usepackage{fancyvrb}
\usepackage{minted}
% \usepackage{enumerate}
%\usepackage{metropolisbsuir} % Don't forget to load this!
\usepackage{molochbsuir} % Don't forget to load this!
\usepackage{tikz}

\usepackage[style=verbose,backend=biber]{biblatex}
\addbibresource{my_ref.bib}

%===============================================================================
%===============================================================================
%\setbeamersize{text margin left=5pt,text margin right=5pt}
%===============================================================================
%\usecolortheme{seahorse}

%\makeatletter
%\setlength{\metropolis@titleseparator@linewidth}{1pt}
%\setlength{\metropolis@progressonsectionpage@linewidth}{4pt}
%\setlength{\metropolis@progressinheadfoot@linewidth}{1pt}
%\makeatother

%\newcommand{\themename}{\textbf{\textsc{metropolis}}\xspace}

\makeatletter
\setbeamertemplate{title page}{
  \begin{minipage}[b]{\textwidth}
    \vfill%
    \ifx\inserttitle\@empty\else\usebeamertemplate*{title}\fi
    \ifx\insertsubtitle\@empty\else\usebeamertemplate*{subtitle}\fi
    \usebeamertemplate*{title separator}
    \ifx\beamer@shortauthor\@empty\else\usebeamertemplate*{author}\fi
    %\vspace{0.5cm} 
  \end{minipage}
  \vfil
  \begin{minipage}[t]{.58\textwidth}
    \vfill%
    \ifx\insertinstitute\@empty\else\usebeamertemplate*{institute}\fi
  \end{minipage}
  %\hfill
  \begin{minipage}[t]{.38\textwidth}
    \vfill
    \ifx\inserttitlegraphic\@empty\else\inserttitlegraphic\fi
  \end{minipage}
  \vfil
  \vspace{10pt}
  \ifx\insertdate\@empty\else\usebeamertemplate{date}\fi
}
\makeatother

\title{%
FPGA реализация нейронной сети прямого распространения для распознавания рукописных чисел}%
%\subtitle{ }
\date{ \tiny 20 ноября, 2024}
\author{{\bfseries Е.А. Кривальцевич}	\and М.И. Вашкевич \\ \texttt{ \footnotesize krivalcevi4.egor@gmail.com,\,\, vashkevich@bsuir.by}}
\institute{
  Белорусский государственный университет\\
  информатики и радиоэлектроники \\
  Кафедра электронных вычислительных средств \\
  \\
  \\
  XIV Международная научная конференции \\
  «Информационные технологии и системы»\\
  Минск, Республика Беларусь
}
\titlegraphic{
    \includegraphics[height=1.3cm]{bsuir-logo.pdf} \\ \vspace{5pt} \\
    \includegraphics[height=1.3cm]{its_logo.jpg}
}

\newcommand{\diag}[1]{	\operatorname{diag} \left( {#1} \right)	}
\newcommand{\ndiag}[1]{	\operatorname{diag} \left\lbrace {#1} \right\rbrace}
\newcommand{\bdiag}[1]{	\operatorname{diag} \left[ {#1} \right]}
\newcommand{\mtinyidx}[1]{	\textsc{\tiny \textit{#1}}}
\providecommand{\abs}[1]{\lvert#1\rvert}
\providecommand{\norm}[1]{\lVert#1\rVert}
\newcommand{\db}[1]{\SI{#1}{\decibel}}
\newcommand{\dbn}[1]{\SI{#1}{}}
\setbeamersize{text margin left=5pt,text margin right=5pt}

\begin{document}

\maketitle
%------------------------------------------------------------------------------------------
%===============================================================================
\begin{frame}
\frametitle{Содержание}

\begin{enumerate}
    \item Прототипирование нейронных сетей на FPGA 
    \item Постановка задачи
    \item Обучение нейронной сети
    \item Аппаратная реализация нейронной сети
    \item Использование PYNQ для прототипирования и тестирования нейронной сети
    \item Описание эксперимента и результаты
\end{enumerate}

\end{frame}
%===============================================================================


\section{Введение}
%===============================================================================
\begin{frame}
\frametitle{Прототипирование нейронных сетей на FPGA}
\begin{columns}[T]
    \column{0.7\textwidth}

    \begin{itemize}
        \item В настоящее время наблюдается интерес к использованию \textbf{кватернионов} 
        при построении \textbf{нейронных сетей} для обработки \textbf{многомерных данных}\footnote{Kusakabe, T., Kouda, N., Isokawa, T., Matsui, N. : A Study of Neural Network Based on Quaternion. Proceeding of SICE Annual Conference (2002) 776–779}
        \item Цветные изображения являются важным примером многомерных данных
        \item Обычно RGB-изображения обрабатываются при помощи 
        сверточных нейронных сетей. Входное изображение 
        интерпретируется, как 3-х мерный тензор
    \end{itemize}

    \column{0.3\textwidth}
    \begin{block}{\centering}
        \vspace{1mm}
        \centering
        \includesvg[height = 0.5\textheight]{RGB_multidim_data.svg}        
    \end{block}
\end{columns}
\end{frame}
%===============================================================================

\begin{frame}
\frametitle{Постановка задачи}

\begin{block}{\centering Цель исследования}                
    \begin{itemize}\small
        \item Получить базовую модель автокодировщика на основе кватернионов
        \item Выяснить дает ли преимущество использование кватернионов
        в задаче сжатия изобаржений при помощи автокодировщика
        \item Оценить преимущества кватернионной нейронной сети над вещественнозначной
        в задаче сжатия изобаржений (MSE, PSNR, SIMM)            
    \end{itemize}
\end{block}
\end{frame}
%===============================================================================
\section{Обучение нейронной сети}
%===============================================================================
\begin{frame}
\frametitle{Архитектура нейронной сети}
\begin{columns}[T]
    \column{0.5\textwidth}
    
    \begin{block}{\centering Архитектура НС}
        \vspace{1mm}
        \centering
        % \includesvg[height = 0.4\textheight]{basic_AE.svg}        
    \end{block}        

    \column{0.5\textwidth}
    \vspace{-2mm}
        
\end{columns}
\end{frame}
%===============================================================================
\section{Аппаратная реализация нейронной сети}
%===============================================================================
\begin{frame}
\frametitle{Аппаратная реализация нейронной сети}

\end{frame}
%===============================================================================
\begin{frame}
    \frametitle{**}

\end{frame}

%===============================================================================
\begin{frame}[t]
    \frametitle{**}

\end{frame}

\section{Использование PYNQ для прототипирования и тестирования нейронной сети}
%===============================================================================
\begin{frame}[t]
    \frametitle{Структурная схема проекта}
    \begin{block}{\centering Архитектура проекта}
        \vspace{1mm}
        \centering \includegraphics[width = 0.9\textwidth]{pics/struct.png}
        % \includesvg[height = 0.4\textheight]{basic_AE.svg}        
    \end{block}   
\end{frame}
%===============================================================================

% \begin{frame}[t]
% \frametitle{Автокодировщик на основе кватернионной НС}

% \begin{block}{\centering QAE -- \emph{quaternion autoencoder}}
%     \centering
% % \includesvg[height = 0.72\textheight]{QAE_scheme_ready.svg} 
% \end{block}
% \end{frame}
%===============================================================================
\section{Эксперимент и результаты}
%===============================================================================
\begin{frame}[t]
\frametitle{Описание эксперимента}
\begin{itemize}
    \item Набор данных MNIST (10 тыс. изображений рукописных цифр $28 \times 28$)
    \item Данные подаются последовательно из процессорной системы
    \item Результаты группируются в виде матриц спутывания
    \item Проведено 15 тестов с различными разрядностями весовых коэффициентов (от 2 до 16)
    \item Составлен график зависимости точности от разрядности
    \item Разложены классы весовых коэффициентов на битовые плоскости
    \item Проанализированы аппаратные затраты
\end{itemize}

\end{frame}

%===============================================================================
\begin{frame}[t]
\frametitle{Результаты}
\begin{columns}
% \hspace{5mm}
\column[t]{0.45\textwidth}
% \begin{center}
\begin{block}{ \centering Матрица спутывания}
    \vspace{3mm}
    \includegraphics[width = 0.815\textwidth]{pics/cm_6q5.jpg} 
\end{block}
% \end{center}
 
\column[t]{0.45\textwidth}
\begin{block}{\centering Точность и затраты блоков LUT/FF}
    \vspace{3mm}
    \includegraphics[width = 0.9\textwidth]{pics/Acc_LUTs_FFs.jpg} 
\end{block}

\end{columns}
\end{frame}

%===============================================================================
\begin{frame}[t]
    \frametitle{Аппаратные затраты}
    \centering
    \begin{table}[h]
        \centering
        \caption{Аппаратные затраты для 5 битного представления коэффициентов}
        \begin{tabular}{|>{\raggedright\arraybackslash}p{4cm}|>{\centering\arraybackslash}p{2.5cm}|>{\centering\arraybackslash}p{2.5cm}|>{\centering\arraybackslash}p{2.5cm}|}
            \hline
            Тип блока & Использовано & Доступно & Соотношение, \% \\
            \hline
            LUT as logic & 2180 & 17600 & 12.39 \\
            \hline
            LUT as memory & 60 & 6000 & 1 \\
            \hline
            Flip Flop & 862 & 35200 & 2.45 \\
            \hline
            RAMB18  & 10 & 120 & 8.33 \\
            \hline
            DSP & 0 & 80 & 0 \\
            \hline
            % BUFG & 1 & 32 & 3.13 \\
            % \hline
        \end{tabular}
    \end{table}
\end{frame}

%===============================================================================
\begin{frame}[t]
\frametitle{Разложение на битовые плоскости}
\begin{columns}
    \hspace{5mm}
    \column[t]{0.5\textwidth}
    \centering 
    \begin{block}{\centering{Битовые плоскости}}
        \vspace{3mm}
        \centering\includegraphics[height = 0.6\textheight]{pics/output1.png} 
    \end{block}
     
    \column[t]{0.5\textwidth}
    \centering 
    \begin{block}{\centering{Зануление части битовых плоскостей}}
        \vspace{3mm}
        \centering\includegraphics[height = 0.6\textheight]{pics/output.png} 
    \end{block}
\end{columns}
\end{frame}



%===============================================================================
\begin{frame}[t]
\frametitle{Выводы}
\begin{itemize}
    \item Рассмотренный эксперимент на основе НС прямого распространения с
    полносвязным слоем показывает, что формат представление 
    весовых данных существенно влияет на точность определения 
    до 5 битной разрядности. Дальнейшее увеличение разрядности 
    не несет значительных изменений в точности.
    \item Предложенная структура НС показывает, что при увеличении
    разрядности наблюдается линейный рост в потреблении LUT и FF блоков FPGA.
\end{itemize}
\end{frame}

%===============================================================================
% \begin{frame}
%     % \printbibliography
% \end{frame}
 

%------------------------------------------------------------------------------------------

%===============================================================================
\end{document}