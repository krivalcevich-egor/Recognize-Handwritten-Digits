%===============================================================================
\begin{frame}
\frametitle{Прототипирование нейронных сетей на FPGA}
% \begin{columns}[T]
    % \column{0.7\textwidth}

    \begin{itemize}
        \item Вычислительной платформой для обучения и
        эксплуатации нейросетевых моделей чаще всего выступают графические процессоры,
        которые содержат множество вычислительных ядер, способных обрабатывать потоки данных
        параллельно. 
        \item FPGA (Field Programmable Gate Array) представляют собой реконфигурируемые вычислительные
        платформы, позволяющие реализовывать параллельно-поточные архитектуры НС.
        \item При реализации НС на базе FPGA появляется
        возможность использовать для представления параметров НС типов данных, обеспечивающих
        различную точность.
    \end{itemize}

    % \column{0.3\textwidth}
    % \begin{block}{\centering}
    %     \vspace{1mm}
    %     \centering
    %     \includesvg[height = 0.5\textheight]{RGB_multidim_data.svg}        
    % \end{block}
% \end{columns}
\end{frame}
%===============================================================================

\begin{frame}
\frametitle{Постановка задачи}

\begin{block}{\centering Цель исследования}                
    \begin{itemize}\small
        \item Получить аппаратно реализованную НС прямого распространения для распознавания рукописных цифр
        \item Выяснить влияние разрядности представления весовых коэффициентов НС на точность определения цифр и аппаратные затрат
        \item Оценить наиболее оптимальную реализацию НС           
    \end{itemize}
\end{block}
\end{frame}
%===============================================================================