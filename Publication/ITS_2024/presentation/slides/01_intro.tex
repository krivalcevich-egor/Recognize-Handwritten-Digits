%===============================================================================
\begin{frame}
\frametitle{Прототипирование нейронных сетей на FPGA}
\begin{columns}[T]
    \column{0.7\textwidth}

    \begin{itemize}
        \item В настоящее время наблюдается интерес к использованию \textbf{кватернионов} 
        при построении \textbf{нейронных сетей} для обработки \textbf{многомерных данных}\footnote{Kusakabe, T., Kouda, N., Isokawa, T., Matsui, N. : A Study of Neural Network Based on Quaternion. Proceeding of SICE Annual Conference (2002) 776–779}
        \item Цветные изображения являются важным примером многомерных данных
        \item Обычно RGB-изображения обрабатываются при помощи 
        сверточных нейронных сетей. Входное изображение 
        интерпретируется, как 3-х мерный тензор
    \end{itemize}

    \column{0.3\textwidth}
    \begin{block}{\centering}
        \vspace{1mm}
        \centering
        \includesvg[height = 0.5\textheight]{RGB_multidim_data.svg}        
    \end{block}
\end{columns}
\end{frame}
%===============================================================================

\begin{frame}
\frametitle{Постановка задачи}

\begin{block}{\centering Цель исследования}                
    \begin{itemize}\small
        \item Получить базовую модель автокодировщика на основе кватернионов
        \item Выяснить дает ли преимущество использование кватернионов
        в задаче сжатия изобаржений при помощи автокодировщика
        \item Оценить преимущества кватернионной нейронной сети над вещественнозначной
        в задаче сжатия изобаржений (MSE, PSNR, SIMM)            
    \end{itemize}
\end{block}
\end{frame}
%===============================================================================