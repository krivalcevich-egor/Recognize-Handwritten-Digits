%===============================================================================
\begin{frame}[t]
\frametitle{Описание эксперимента}
% \begin{itemize}
%     \item Набор данных CIFAR-10 (60 тыс. RGB-изображения $32 \times 32$)
%     \item В кодере использовалась функция активации ReLU, в декодере -- логистический сигмоид
%     \item Обучались автокодировщики с различным размером внутреннего слоя
%      $N_{hidden} = \{2, 4, 8, 16, 32, 64, 128, 256, 512, 1024, 2048\}$
%     \item Инициализация весов выполнялась методом Хе
%     \item Целевая функция -- $MSE$
%     \item Оптимизация производилось с использованием алгоритма Adam 
%     (скорость обучения $\eta =10^{−3}$, число эпох -- 50, размер батча -- 256)
%     \item Для оценки качества декодирования изображений использовались 
%     метрики $MSE$, $PSNR$ и $SIMM$
% \end{itemize}

\end{frame}

%===============================================================================
\begin{frame}[t]
\frametitle{Результаты}
\begin{columns}
\column[t]{0.33\textwidth}
\begin{block}{MSE}
    \vspace{3mm}
    % \includegraphics[width = \textwidth]{MSE_result_vert.jpg} 
\end{block}
 
\column[t]{0.33\textwidth}
\begin{block}{PSNR}
    \vspace{3mm}
    % \includegraphics[width = \textwidth]{PSNR_result_vert.jpg} 
\end{block}

\column[t]{0.33\textwidth}
\begin{block}{SIMM}
    \vspace{3mm}
    % \includegraphics[width = \textwidth]{SIMM_result_vert.jpg} 
\end{block}

\end{columns}
\end{frame}

%===============================================================================
\begin{frame}[t]
\frametitle{Результат работы автокодировщика RAE-2048}
\centering
% \includegraphics[height = 0.8\textheight]{RAE_h2048_results_worsed_MSE.jpg} 
\end{frame}

%===============================================================================
\begin{frame}[t]
    \frametitle{Результат работы автокодировщика QAE-2048}
    \centering
    % \includegraphics[height = 0.8\textheight]{QAE_h2048_results_worsed_MSE.jpg} 
\end{frame}

%===============================================================================
\begin{frame}[t]
\frametitle{Сравнение полученных результатов}
\begin{itemize}
    \item Полученные автокодировщики на основе кватернионной НС сравнивались с 
    другими вещественнозначным автокодировщиком, имеющим архитектуру 
    <<бутылочное горлышко>>\footcite{watkins2018image}.
    \item Автокодировщик\footnotemark[1], использует НС с 
    двумя слоями: первый сверхточный, а второй полносвязный и 
    обеспечивает степень сжатия 2:1, т.е. внутренний слой содержал 
    в два раза меньше элементов, чем входной и выходной.
    \item У автокодировщика\footnotemark[1]  
    индекс структурной схожести (SIMM) равен {\bf 0,905}. 
    \item Полученный в данной работе автокодировщик QAE-1024 обеспечивает 
    сжатие 3:1 и имеет SIMM равный \textbf{0,922}.
\end{itemize}

\end{frame}

%===============================================================================
\begin{frame}[t]
\frametitle{Выводы}
\begin{itemize}
    \item Рассмотренный вычислительный эксперимент на основе 
    полносвязного автокодировщика показывает, что представление 
    скоррелированных данных, таких как цветные RGB-изображения, 
    в алгебре кватернионов позволяет лучше учитывать характер 
    исходных данных. 
    \item Предложенные автокодировщики, имеющие различную степень 
    сжатия, позволяют получить более высокие значения объективных 
    характеристик декодирования цветных изображений по сравнению с 
    аналогичными вещественнозначными автокодировщиками (PSNR в среднем 
    выше на 3,85 дБ, SSIM на 0,18).
\end{itemize}
\end{frame}

%===============================================================================
% \begin{frame}
%     % \printbibliography
% \end{frame}
